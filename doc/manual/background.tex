We refer to the following articles for a background of the methods and algorithms implemented in \dftfe. \\

\noindent 1. P. Motamarri, S. Das, S. Rudraraju, K. Ghosh, D. Davydov, V. Gavini, DFT-FE–A massively parallel adaptive finite-element code for large-scale density functional theory calculations, \emph{Comput. Phys. Commun.} 246, 106853 (2020).\\

\noindent 1. P. Motamarri, M.R. Nowak, K. Leiter , J. Knap, V. Gavini, Higher-order adaptive finite-element methods for Kohn-Sham density functional theory, \emph{J. Comput. Phys.} 253, 308-343 (2013).\\
 
\noindent 3. P. Motamarri, V. Gavini,  Configurational forces in electronic structure calculations using Kohn-Sham density functional theory, \emph{Phys. Rev. B} 97 165132 (2018).\\

\noindent In addition, below are some useful references on finite element method and some online resources that provide a background of finite elements and their application to the solution of partial differential equations.\\

\noindent 1. T.J.R. Hughes, The finite element method: linear static and dynamic finite element analysis, Dover Publication, 2000.\\

\noindent 2. K.-J. Bathe, Finite element procedures, Klaus-J\"{u}rgen Bathe, 2014.\\

\noindent 3. The finite element method for problems in physics, online course by Krishna Garikipati. \href{https://www.coursera.org/learn/finite-element-method}{Link}\\

\noindent 4. Online lectures on ``Finite element methods in scientific computing" by Wolfgang Bangerth. \href{http://www.math.colostate.edu/~bangerth/videos.html}{Link}\\ 

