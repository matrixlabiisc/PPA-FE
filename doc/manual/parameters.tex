\subsection{Global parameters}
\label{parameters:global}


\begin{itemize}
\item {\it Parameter name:} {\tt REPRODUCIBLE OUTPUT}
\phantomsection\label{parameters:REPRODUCIBLE OUTPUT}
\label{parameters:REPRODUCIBLE_20OUTPUT}


\index[prmindex]{REPRODUCIBLE OUTPUT}
\index[prmindexfull]{REPRODUCIBLE OUTPUT}


{\it Default:} false


{\it Description:} [Developer] Limit output to what is reproducible, i.e. don't print timing or absolute paths. This parameter is only used for testing purposes.


{\it Possible values:} A boolean value (true or false)
\item {\it Parameter name:} {\tt VERBOSITY}
\phantomsection\label{parameters:VERBOSITY}


\index[prmindex]{VERBOSITY}
\index[prmindexfull]{VERBOSITY}


{\it Default:} 1


{\it Description:} [Standard] Parameter to control verbosity of terminal output. Ranges from 1 for low, 2 for medium (prints eigenvalues and fractional occupancies at the end of each ground-state solve), 3 for high (prints eigenvalues and fractional occupancies at the end of each self-consistent field iteration), and 4 for very high, which is only meant for code development purposes. VERBOSITY=0 is only used for unit testing and shouldn't be used by standard users.


{\it Possible values:} An integer $n$ such that $0\leq n \leq 4$
\item {\it Parameter name:} {\tt WRITE DENSITY}
\phantomsection\label{parameters:WRITE DENSITY}
\label{parameters:WRITE_20DENSITY}


\index[prmindex]{WRITE DENSITY}
\index[prmindexfull]{WRITE DENSITY}


{\it Default:} false


{\it Description:} [Standard] Writes DFT ground state electron-density solution fields (FEM mesh nodal values) to densityOutput.vtu file for visualization purposes. The electron-density solution field in densityOutput.vtu is named density. In case of spin-polarized calculation, two additional solution fields- density\_0 and density\_1 are also written where 0 and 1 denote the spin indices. In the case of geometry optimization, the electron-density corresponding to the last ground-state solve is written. Default: false.


{\it Possible values:} A boolean value (true or false)
\item {\it Parameter name:} {\tt WRITE WFC}
\phantomsection\label{parameters:WRITE WFC}
\label{parameters:WRITE_20WFC}


\index[prmindex]{WRITE WFC}
\index[prmindexfull]{WRITE WFC}


{\it Default:} false


{\it Description:} [Standard] Writes DFT ground state wavefunction solution fields (FEM mesh nodal values) to wfcOutput.vtu file for visualization purposes. The wavefunction solution fields in wfcOutput.vtu are named wfc\_s\_k\_i in case of spin-polarized calculations and wfc\_k\_i otherwise, where s denotes the spin index (0 or 1), k denotes the k point index starting from 0, and i denotes the Kohn-Sham wavefunction index starting from 0. In the case of geometry optimization, the wavefunctions corresponding to the last ground-state solve are written.  Default: false.


{\it Possible values:} A boolean value (true or false)
\end{itemize}



\subsection{Parameters in section \tt Boundary conditions}
\label{parameters:Boundary_20conditions}

\begin{itemize}
\item {\it Parameter name:} {\tt PERIODIC1}
\phantomsection\label{parameters:Boundary conditions/PERIODIC1}
\label{parameters:Boundary_20conditions/PERIODIC1}


\index[prmindex]{PERIODIC1}
\index[prmindexfull]{Boundary conditions!PERIODIC1}


{\it Default:} false


{\it Description:} [Standard] Periodicity along the first domain bounding vector.


{\it Possible values:} A boolean value (true or false)
\item {\it Parameter name:} {\tt PERIODIC2}
\phantomsection\label{parameters:Boundary conditions/PERIODIC2}
\label{parameters:Boundary_20conditions/PERIODIC2}


\index[prmindex]{PERIODIC2}
\index[prmindexfull]{Boundary conditions!PERIODIC2}


{\it Default:} false


{\it Description:} [Standard] Periodicity along the second domain bounding vector.


{\it Possible values:} A boolean value (true or false)
\item {\it Parameter name:} {\tt PERIODIC3}
\phantomsection\label{parameters:Boundary conditions/PERIODIC3}
\label{parameters:Boundary_20conditions/PERIODIC3}


\index[prmindex]{PERIODIC3}
\index[prmindexfull]{Boundary conditions!PERIODIC3}


{\it Default:} false


{\it Description:} [Standard] Periodicity along the third domain bounding vector.


{\it Possible values:} A boolean value (true or false)
\item {\it Parameter name:} {\tt SELF POTENTIAL RADIUS}
\phantomsection\label{parameters:Boundary conditions/SELF POTENTIAL RADIUS}
\label{parameters:Boundary_20conditions/SELF_20POTENTIAL_20RADIUS}


\index[prmindex]{SELF POTENTIAL RADIUS}
\index[prmindexfull]{Boundary conditions!SELF POTENTIAL RADIUS}


{\it Default:} 0.0


{\it Description:} [Advanced] The radius (in a.u) of the ball around an atom in which self-potential of the associated nuclear charge is solved. For the default value of 0.0, the radius value is automatically determined to accommodate the largest radius possible for the given finite element mesh. The default approach works for most problems.


{\it Possible values:} A floating point number $v$ such that $0 \leq v \leq 10$
\end{itemize}

\subsection{Parameters in section \tt Brillouin zone k point sampling options}
\label{parameters:Brillouin_20zone_20k_20point_20sampling_20options}

\begin{itemize}
\item {\it Parameter name:} {\tt USE GROUP SYMMETRY}
\phantomsection\label{parameters:Brillouin zone k point sampling options/USE GROUP SYMMETRY}
\label{parameters:Brillouin_20zone_20k_20point_20sampling_20options/USE_20GROUP_20SYMMETRY}


\index[prmindex]{USE GROUP SYMMETRY}
\index[prmindexfull]{Brillouin zone k point sampling options!USE GROUP SYMMETRY}


{\it Default:} false


{\it Description:} [Standard] Flag to control the use of point group symmetries. Currently this feature cannot be used if ION FORCE or CELL STRESS input parameters are set to true.


{\it Possible values:} A boolean value (true or false)
\item {\it Parameter name:} {\tt USE TIME REVERSAL SYMMETRY}
\phantomsection\label{parameters:Brillouin zone k point sampling options/USE TIME REVERSAL SYMMETRY}
\label{parameters:Brillouin_20zone_20k_20point_20sampling_20options/USE_20TIME_20REVERSAL_20SYMMETRY}


\index[prmindex]{USE TIME REVERSAL SYMMETRY}
\index[prmindexfull]{Brillouin zone k point sampling options!USE TIME REVERSAL SYMMETRY}


{\it Default:} false


{\it Description:} [Standard] Flag to control the use of time reversal symmetry.


{\it Possible values:} A boolean value (true or false)
\item {\it Parameter name:} {\tt kPOINT RULE FILE}
\phantomsection\label{parameters:Brillouin zone k point sampling options/kPOINT RULE FILE}
\label{parameters:Brillouin_20zone_20k_20point_20sampling_20options/kPOINT_20RULE_20FILE}


\index[prmindex]{kPOINT RULE FILE}
\index[prmindexfull]{Brillouin zone k point sampling options!kPOINT RULE FILE}


{\it Default:} 


{\it Description:} [Developer] File specifying the k-Point rule to sample Brillouin zone. CAUTION: This option is recommended only for postprocessing, for example band structure calculation. To set k point rule for DFT solve use the Monkhorst-Pack (MP) grid generation.


{\it Possible values:} Any string
\end{itemize}



\subsection{Parameters in section \tt Brillouin zone k point sampling options/Monkhorst-Pack (MP) grid generation}
\label{parameters:Brillouin_20zone_20k_20point_20sampling_20options/Monkhorst_2dPack_20_28MP_29_20grid_20generation}

\begin{itemize}
\item {\it Parameter name:} {\tt SAMPLING POINTS 1}
\phantomsection\label{parameters:Brillouin zone k point sampling options/Monkhorst_2dPack _28MP_29 grid generation/SAMPLING POINTS 1}
\label{parameters:Brillouin_20zone_20k_20point_20sampling_20options/Monkhorst_2dPack_20_28MP_29_20grid_20generation/SAMPLING_20POINTS_201}


\index[prmindex]{SAMPLING POINTS 1}
\index[prmindexfull]{Brillouin zone k point sampling options!Monkhorst-Pack (MP) grid generation!SAMPLING POINTS 1}


{\it Default:} 1


{\it Description:} [Standard] Number of Monkhorst-Pack grid points to be used along reciprocal lattice vector 1.


{\it Possible values:} An integer $n$ such that $1\leq n \leq 1000$
\item {\it Parameter name:} {\tt SAMPLING POINTS 2}
\phantomsection\label{parameters:Brillouin zone k point sampling options/Monkhorst_2dPack _28MP_29 grid generation/SAMPLING POINTS 2}
\label{parameters:Brillouin_20zone_20k_20point_20sampling_20options/Monkhorst_2dPack_20_28MP_29_20grid_20generation/SAMPLING_20POINTS_202}


\index[prmindex]{SAMPLING POINTS 2}
\index[prmindexfull]{Brillouin zone k point sampling options!Monkhorst-Pack (MP) grid generation!SAMPLING POINTS 2}


{\it Default:} 1


{\it Description:} [Standard] Number of Monkhorst-Pack grid points to be used along reciprocal lattice vector 2.


{\it Possible values:} An integer $n$ such that $1\leq n \leq 1000$
\item {\it Parameter name:} {\tt SAMPLING POINTS 3}
\phantomsection\label{parameters:Brillouin zone k point sampling options/Monkhorst_2dPack _28MP_29 grid generation/SAMPLING POINTS 3}
\label{parameters:Brillouin_20zone_20k_20point_20sampling_20options/Monkhorst_2dPack_20_28MP_29_20grid_20generation/SAMPLING_20POINTS_203}


\index[prmindex]{SAMPLING POINTS 3}
\index[prmindexfull]{Brillouin zone k point sampling options!Monkhorst-Pack (MP) grid generation!SAMPLING POINTS 3}


{\it Default:} 1


{\it Description:} [Standard] Number of Monkhorst-Pack grid points to be used along reciprocal lattice vector 3.


{\it Possible values:} An integer $n$ such that $1\leq n \leq 1000$
\item {\it Parameter name:} {\tt SAMPLING SHIFT 1}
\phantomsection\label{parameters:Brillouin zone k point sampling options/Monkhorst_2dPack _28MP_29 grid generation/SAMPLING SHIFT 1}
\label{parameters:Brillouin_20zone_20k_20point_20sampling_20options/Monkhorst_2dPack_20_28MP_29_20grid_20generation/SAMPLING_20SHIFT_201}


\index[prmindex]{SAMPLING SHIFT 1}
\index[prmindexfull]{Brillouin zone k point sampling options!Monkhorst-Pack (MP) grid generation!SAMPLING SHIFT 1}


{\it Default:} 0


{\it Description:} [Standard] If fractional shifting to be used (0 for no shift, 1 for shift) along reciprocal lattice vector 1.


{\it Possible values:} An integer $n$ such that $0\leq n \leq 1$
\item {\it Parameter name:} {\tt SAMPLING SHIFT 2}
\phantomsection\label{parameters:Brillouin zone k point sampling options/Monkhorst_2dPack _28MP_29 grid generation/SAMPLING SHIFT 2}
\label{parameters:Brillouin_20zone_20k_20point_20sampling_20options/Monkhorst_2dPack_20_28MP_29_20grid_20generation/SAMPLING_20SHIFT_202}


\index[prmindex]{SAMPLING SHIFT 2}
\index[prmindexfull]{Brillouin zone k point sampling options!Monkhorst-Pack (MP) grid generation!SAMPLING SHIFT 2}


{\it Default:} 0


{\it Description:} [Standard] If fractional shifting to be used (0 for no shift, 1 for shift) along reciprocal lattice vector 2.


{\it Possible values:} An integer $n$ such that $0\leq n \leq 1$
\item {\it Parameter name:} {\tt SAMPLING SHIFT 3}
\phantomsection\label{parameters:Brillouin zone k point sampling options/Monkhorst_2dPack _28MP_29 grid generation/SAMPLING SHIFT 3}
\label{parameters:Brillouin_20zone_20k_20point_20sampling_20options/Monkhorst_2dPack_20_28MP_29_20grid_20generation/SAMPLING_20SHIFT_203}


\index[prmindex]{SAMPLING SHIFT 3}
\index[prmindexfull]{Brillouin zone k point sampling options!Monkhorst-Pack (MP) grid generation!SAMPLING SHIFT 3}


{\it Default:} 0


{\it Description:} [Standard] If fractional shifting to be used (0 for no shift, 1 for shift) along reciprocal lattice vector 3.


{\it Possible values:} An integer $n$ such that $0\leq n \leq 1$
\end{itemize}

\subsection{Parameters in section \tt Checkpointing and Restart}
\label{parameters:Checkpointing_20and_20Restart}

\begin{itemize}
\item {\it Parameter name:} {\tt CHK TYPE}
\phantomsection\label{parameters:Checkpointing and Restart/CHK TYPE}
\label{parameters:Checkpointing_20and_20Restart/CHK_20TYPE}


\index[prmindex]{CHK TYPE}
\index[prmindexfull]{Checkpointing and Restart!CHK TYPE}


{\it Default:} 0


{\it Description:} [Standard] Checkpoint type, 0 (do not create any checkpoint), 1 (create checkpoint for geometry optimization restart if either ION OPT or CELL OPT is set to true. Currently, checkpointing and restart framework does not work if both ION OPT and CELL OPT are set to true simultaneously- the code will throw an error if attempted.), 2 (create checkpoint for scf restart. Currently, this option cannot be used if geometry optimization is being performed. The code will throw an error if this option is used in conjunction with geometry optimization.)


{\it Possible values:} An integer $n$ such that $0\leq n \leq 2$
\item {\it Parameter name:} {\tt RESTART FROM CHK}
\phantomsection\label{parameters:Checkpointing and Restart/RESTART FROM CHK}
\label{parameters:Checkpointing_20and_20Restart/RESTART_20FROM_20CHK}


\index[prmindex]{RESTART FROM CHK}
\index[prmindexfull]{Checkpointing and Restart!RESTART FROM CHK}


{\it Default:} false


{\it Description:} [Standard] Boolean parameter specifying if the current job reads from a checkpoint. The nature of the restart corresponds to the CHK TYPE parameter. Hence, the checkpoint being read must have been created using the CHK TYPE parameter before using this option. RESTART FROM CHK is always false for CHK TYPE 0.


{\it Possible values:} A boolean value (true or false)
\end{itemize}

\subsection{Parameters in section \tt DFT functional parameters}
\label{parameters:DFT_20functional_20parameters}

\begin{itemize}
\item {\it Parameter name:} {\tt EXCHANGE CORRELATION TYPE}
\phantomsection\label{parameters:DFT functional parameters/EXCHANGE CORRELATION TYPE}
\label{parameters:DFT_20functional_20parameters/EXCHANGE_20CORRELATION_20TYPE}


\index[prmindex]{EXCHANGE CORRELATION TYPE}
\index[prmindexfull]{DFT functional parameters!EXCHANGE CORRELATION TYPE}


{\it Default:} 1


{\it Description:} [Standard] Parameter specifying the type of exchange-correlation to be used: 1(LDA: Perdew Zunger Ceperley Alder correlation with Slater Exchange[PRB. 23, 5048 (1981)]), 2(LDA: Perdew-Wang 92 functional with Slater Exchange [PRB. 45, 13244 (1992)]), 3(LDA: Vosko, Wilk \& Nusair with Slater Exchange[Can. J. Phys. 58, 1200 (1980)]), 4(GGA: Perdew-Burke-Ernzerhof functional [PRL. 77, 3865 (1996)]).


{\it Possible values:} An integer $n$ such that $1\leq n \leq 4$
\item {\it Parameter name:} {\tt PSEUDOPOTENTIAL CALCULATION}
\phantomsection\label{parameters:DFT functional parameters/PSEUDOPOTENTIAL CALCULATION}
\label{parameters:DFT_20functional_20parameters/PSEUDOPOTENTIAL_20CALCULATION}


\index[prmindex]{PSEUDOPOTENTIAL CALCULATION}
\index[prmindexfull]{DFT functional parameters!PSEUDOPOTENTIAL CALCULATION}


{\it Default:} true


{\it Description:} [Standard] Boolean Parameter specifying whether pseudopotential DFT calculation needs to be performed. For all-electron DFT calculation set to false.


{\it Possible values:} A boolean value (true or false)
\item {\it Parameter name:} {\tt PSEUDOPOTENTIAL FILE NAMES LIST}
\phantomsection\label{parameters:DFT functional parameters/PSEUDOPOTENTIAL FILE NAMES LIST}
\label{parameters:DFT_20functional_20parameters/PSEUDOPOTENTIAL_20FILE_20NAMES_20LIST}


\index[prmindex]{PSEUDOPOTENTIAL FILE NAMES LIST}
\index[prmindexfull]{DFT functional parameters!PSEUDOPOTENTIAL FILE NAMES LIST}


{\it Default:} 


{\it Description:} [Standard] Pseudopotential file. This file contains the list of pseudopotential file names in UPF format corresponding to the atoms involved in the calculations. UPF version 2.0 or greater and norm-conserving pseudopotentials(ONCV and Troullier Martins) in UPF format are only accepted. File format (example for two atoms Mg(z=12), Al(z=13)): 12 filename1.upf(row1), 13 filename2.upf (row2). Important Note: ONCV pseudopotentials data base in UPF format can be downloaded from http://www.quantum-simulation.org/potentials/sg15\_oncv.  Troullier-Martins pseudopotentials in UPF format can be downloaded from http://www.quantum-espresso.org/pseudopotentials/fhi-pp-from-abinit-web-site.


{\it Possible values:} Any string
\item {\it Parameter name:} {\tt PSEUDO TESTS FLAG}
\phantomsection\label{parameters:DFT functional parameters/PSEUDO TESTS FLAG}
\label{parameters:DFT_20functional_20parameters/PSEUDO_20TESTS_20FLAG}


\index[prmindex]{PSEUDO TESTS FLAG}
\index[prmindexfull]{DFT functional parameters!PSEUDO TESTS FLAG}


{\it Default:} false


{\it Description:} [Developer] Boolean parameter specifying the explicit path of pseudopotential upf format files used for ctests


{\it Possible values:} A boolean value (true or false)
\item {\it Parameter name:} {\tt SPIN POLARIZATION}
\phantomsection\label{parameters:DFT functional parameters/SPIN POLARIZATION}
\label{parameters:DFT_20functional_20parameters/SPIN_20POLARIZATION}


\index[prmindex]{SPIN POLARIZATION}
\index[prmindexfull]{DFT functional parameters!SPIN POLARIZATION}


{\it Default:} 0


{\it Description:} [Standard] Spin polarization: 0 for no spin polarization and 1 for collinear spin polarization calculation. Default option is 0.


{\it Possible values:} An integer $n$ such that $0\leq n \leq 1$
\item {\it Parameter name:} {\tt START MAGNETIZATION}
\phantomsection\label{parameters:DFT functional parameters/START MAGNETIZATION}
\label{parameters:DFT_20functional_20parameters/START_20MAGNETIZATION}


\index[prmindex]{START MAGNETIZATION}
\index[prmindexfull]{DFT functional parameters!START MAGNETIZATION}


{\it Default:} 0.0


{\it Description:} [Standard] Starting magnetization to be used for spin-polarized DFT calculations (must be between -0.5 and +0.5). Corresponding magnetization per simulation domain will be (2 x START MAGNETIZATION x Number of electrons) a.u. 


{\it Possible values:} A floating point number $v$ such that $-0.5 \leq v \leq 0.5$
\end{itemize}

\subsection{Parameters in section \tt Finite element mesh parameters}
\label{parameters:Finite_20element_20mesh_20parameters}

\begin{itemize}
\item {\it Parameter name:} {\tt MESH FILE}
\phantomsection\label{parameters:Finite element mesh parameters/MESH FILE}
\label{parameters:Finite_20element_20mesh_20parameters/MESH_20FILE}


\index[prmindex]{MESH FILE}
\index[prmindexfull]{Finite element mesh parameters!MESH FILE}


{\it Default:} 


{\it Description:} [Developer] External mesh file path. If nothing is given auto mesh generation is performed. The option is only for testing purposes.


{\it Possible values:} Any string
\item {\it Parameter name:} {\tt POLYNOMIAL ORDER}
\phantomsection\label{parameters:Finite element mesh parameters/POLYNOMIAL ORDER}
\label{parameters:Finite_20element_20mesh_20parameters/POLYNOMIAL_20ORDER}


\index[prmindex]{POLYNOMIAL ORDER}
\index[prmindexfull]{Finite element mesh parameters!POLYNOMIAL ORDER}


{\it Default:} 4


{\it Description:} [Standard] The degree of the finite-element interpolating polynomial. Default value is 4. POLYNOMIAL ORDER= 4 or 5 is usually a good choice for most pseudopotential as well as all-electron problems.


{\it Possible values:} An integer $n$ such that $1\leq n \leq 12$
\end{itemize}



\subsection{Parameters in section \tt Finite element mesh parameters/Auto mesh generation parameters}
\label{parameters:Finite_20element_20mesh_20parameters/Auto_20mesh_20generation_20parameters}

\begin{itemize}
\item {\it Parameter name:} {\tt ATOM BALL RADIUS}
\phantomsection\label{parameters:Finite element mesh parameters/Auto mesh generation parameters/ATOM BALL RADIUS}
\label{parameters:Finite_20element_20mesh_20parameters/Auto_20mesh_20generation_20parameters/ATOM_20BALL_20RADIUS}


\index[prmindex]{ATOM BALL RADIUS}
\index[prmindexfull]{Finite element mesh parameters!Auto mesh generation parameters!ATOM BALL RADIUS}


{\it Default:} 2.0


{\it Description:} [Advanced] Radius of ball enclosing every atom, inside which the mesh size is set close to MESH SIZE AROUND ATOM. The default value of 2.0 is good enough for most cases. On rare cases, where the nonlocal pseudopotential projectors have a compact support beyond 2.0, a slightly larger ATOM BALL RADIUS between 2.0 to 2.5 may be required. Standard users do not need to tune this parameter. Units: a.u.


{\it Possible values:} A floating point number $v$ such that $0 \leq v \leq 20$
\item {\it Parameter name:} {\tt BASE MESH SIZE}
\phantomsection\label{parameters:Finite element mesh parameters/Auto mesh generation parameters/BASE MESH SIZE}
\label{parameters:Finite_20element_20mesh_20parameters/Auto_20mesh_20generation_20parameters/BASE_20MESH_20SIZE}


\index[prmindex]{BASE MESH SIZE}
\index[prmindexfull]{Finite element mesh parameters!Auto mesh generation parameters!BASE MESH SIZE}


{\it Default:} 0.0


{\it Description:} [Advanced] Mesh size of the base mesh on which refinement is performed. For the default value of 0.0, a heuristically determined base mesh size is used, which is good enough for most cases. Standard users do not need to tune this parameter. Units: a.u.


{\it Possible values:} A floating point number $v$ such that $0 \leq v \leq 20$
\item {\it Parameter name:} {\tt MESH SIZE AROUND ATOM}
\phantomsection\label{parameters:Finite element mesh parameters/Auto mesh generation parameters/MESH SIZE AROUND ATOM}
\label{parameters:Finite_20element_20mesh_20parameters/Auto_20mesh_20generation_20parameters/MESH_20SIZE_20AROUND_20ATOM}


\index[prmindex]{MESH SIZE AROUND ATOM}
\index[prmindexfull]{Finite element mesh parameters!Auto mesh generation parameters!MESH SIZE AROUND ATOM}


{\it Default:} 0.8


{\it Description:} [Standard] Mesh size in a ball of radius ATOM BALL RADIUS around every atom. For pseudopotential calculations, a value between 0.5 to 1.0 is usually a good choice. For all-electron calculations, a value between 0.1 to 0.3 would be a good starting choice. In most cases, MESH SIZE AROUND ATOM is the only parameter to be tuned to achieve the desired accuracy in energy and forces with respect to the mesh refinement. Units: a.u.


{\it Possible values:} A floating point number $v$ such that $0.0001 \leq v \leq 10$
\item {\it Parameter name:} {\tt MESH SIZE AT ATOM}
\phantomsection\label{parameters:Finite element mesh parameters/Auto mesh generation parameters/MESH SIZE AT ATOM}
\label{parameters:Finite_20element_20mesh_20parameters/Auto_20mesh_20generation_20parameters/MESH_20SIZE_20AT_20ATOM}


\index[prmindex]{MESH SIZE AT ATOM}
\index[prmindexfull]{Finite element mesh parameters!Auto mesh generation parameters!MESH SIZE AT ATOM}


{\it Default:} 0.0


{\it Description:} [Advanced] Mesh size of the finite elements in the immediate vicinity of the atom. For the default value of 0.0, a heuristically determined MESH SIZE AT ATOM is used, which is good enough for most cases. Standard users do not need to tune this parameter. Units: a.u.


{\it Possible values:} A floating point number $v$ such that $0 \leq v \leq 10$
\end{itemize}

\subsection{Parameters in section \tt Geometry}
\label{parameters:Geometry}

\begin{itemize}
\item {\it Parameter name:} {\tt ATOMIC COORDINATES FILE}
\phantomsection\label{parameters:Geometry/ATOMIC COORDINATES FILE}
\label{parameters:Geometry/ATOMIC_20COORDINATES_20FILE}


\index[prmindex]{ATOMIC COORDINATES FILE}
\index[prmindexfull]{Geometry!ATOMIC COORDINATES FILE}


{\it Default:} 


{\it Description:} [Standard] Atomic-coordinates input file name. For fully non-periodic domain give Cartesian coordinates of the atoms (in a.u) with respect to origin at the center of the domain. For periodic and semi-periodic domain give fractional coordinates of atoms. File format (example for two atoms): Atom1-atomic-charge Atom1-valence-charge x1 y1 z1 (row1), Atom2-atomic-charge Atom2-valence-charge x2 y2 z2 (row2). The number of rows must be equal to NATOMS, and number of unique atoms must be equal to NATOM TYPES.


{\it Possible values:} Any string
\item {\it Parameter name:} {\tt DOMAIN VECTORS FILE}
\phantomsection\label{parameters:Geometry/DOMAIN VECTORS FILE}
\label{parameters:Geometry/DOMAIN_20VECTORS_20FILE}


\index[prmindex]{DOMAIN VECTORS FILE}
\index[prmindexfull]{Geometry!DOMAIN VECTORS FILE}


{\it Default:} 


{\it Description:} [Standard] Domain vectors input file name. Domain vectors are the vectors bounding the three edges of the 3D parallelepiped computational domain. File format: v1x v1y v1z (row1), v2x v2y v2z (row2), v3x v3y v3z (row3). Units: a.u. CAUTION: please ensure that the domain vectors form a right-handed coordinate system i.e. dotProduct(crossProduct(v1,v2),v3)>0. Domain vectors are the typical lattice vectors in a fully periodic calculation.


{\it Possible values:} Any string
\item {\it Parameter name:} {\tt NATOMS}
\phantomsection\label{parameters:Geometry/NATOMS}


\index[prmindex]{NATOMS}
\index[prmindexfull]{Geometry!NATOMS}


{\it Default:} 0


{\it Description:} [Standard] Total number of atoms. This parameter requires a mandatory non-zero input which is equal to the number of rows in the file passed to ATOMIC COORDINATES FILE.


{\it Possible values:} An integer $n$ such that $0\leq n \leq 2147483647$
\item {\it Parameter name:} {\tt NATOM TYPES}
\phantomsection\label{parameters:Geometry/NATOM TYPES}
\label{parameters:Geometry/NATOM_20TYPES}


\index[prmindex]{NATOM TYPES}
\index[prmindexfull]{Geometry!NATOM TYPES}


{\it Default:} 0


{\it Description:} [Standard] Total number of atom types. This parameter requires a mandatory non-zero input which is equal to the number of unique atom types in the file passed to ATOMIC COORDINATES FILE.


{\it Possible values:} An integer $n$ such that $0\leq n \leq 2147483647$
\end{itemize}



\subsection{Parameters in section \tt Geometry/Optimization}
\label{parameters:Geometry/Optimization}

\begin{itemize}
\item {\it Parameter name:} {\tt CELL CONSTRAINT TYPE}
\phantomsection\label{parameters:Geometry/Optimization/CELL CONSTRAINT TYPE}
\label{parameters:Geometry/Optimization/CELL_20CONSTRAINT_20TYPE}


\index[prmindex]{CELL CONSTRAINT TYPE}
\index[prmindexfull]{Geometry!Optimization!CELL CONSTRAINT TYPE}


{\it Default:} 12


{\it Description:} [Standard] Cell relaxation constraint type, 1 (isotropic shape-fixed volume optimization), 2 (volume-fixed shape optimization), 3 (relax along domain vector component v1x), 4 (relax along domain vector component v2x), 5 (relax along domain vector component v3x), 6 (relax along domain vector components v2x and v3x), 7 (relax along domain vector components v1x and v3x), 8 (relax along domain vector components v1x and v2x), 9 (volume optimization- relax along domain vector components v1x, v2x and v3x), 10 (2D - relax along x and y components), 11(2D- relax only x and y components with inplane area fixed), 12(relax all domain vector components), 13 automatically decides the constraints based on boundary conditions. CAUTION: A majority of these options only make sense in an orthorhombic cell geometry.


{\it Possible values:} An integer $n$ such that $1\leq n \leq 13$
\item {\it Parameter name:} {\tt CELL OPT}
\phantomsection\label{parameters:Geometry/Optimization/CELL OPT}
\label{parameters:Geometry/Optimization/CELL_20OPT}


\index[prmindex]{CELL OPT}
\index[prmindexfull]{Geometry!Optimization!CELL OPT}


{\it Default:} false


{\it Description:} [Standard] Boolean parameter specifying if cell needs to be relaxed to achieve zero stress


{\it Possible values:} A boolean value (true or false)
\item {\it Parameter name:} {\tt CELL STRESS}
\phantomsection\label{parameters:Geometry/Optimization/CELL STRESS}
\label{parameters:Geometry/Optimization/CELL_20STRESS}


\index[prmindex]{CELL STRESS}
\index[prmindexfull]{Geometry!Optimization!CELL STRESS}


{\it Default:} false


{\it Description:} [Standard] Boolean parameter specifying if cell stress needs to be computed. Automatically set to true if CELL OPT is true.


{\it Possible values:} A boolean value (true or false)
\item {\it Parameter name:} {\tt FORCE TOL}
\phantomsection\label{parameters:Geometry/Optimization/FORCE TOL}
\label{parameters:Geometry/Optimization/FORCE_20TOL}


\index[prmindex]{FORCE TOL}
\index[prmindexfull]{Geometry!Optimization!FORCE TOL}


{\it Default:} 1e-4


{\it Description:} [Standard] Sets the tolerance on the maximum force (in a.u.) on an atom during atomic relaxation, when the atoms are considered to be relaxed.


{\it Possible values:} A floating point number $v$ such that $0 \leq v \leq 1$
\item {\it Parameter name:} {\tt ION FORCE}
\phantomsection\label{parameters:Geometry/Optimization/ION FORCE}
\label{parameters:Geometry/Optimization/ION_20FORCE}


\index[prmindex]{ION FORCE}
\index[prmindexfull]{Geometry!Optimization!ION FORCE}


{\it Default:} false


{\it Description:} [Standard] Boolean parameter specifying if atomic forces are to be computed. Automatically set to true if ION OPT is true.


{\it Possible values:} A boolean value (true or false)
\item {\it Parameter name:} {\tt ION OPT}
\phantomsection\label{parameters:Geometry/Optimization/ION OPT}
\label{parameters:Geometry/Optimization/ION_20OPT}


\index[prmindex]{ION OPT}
\index[prmindexfull]{Geometry!Optimization!ION OPT}


{\it Default:} false


{\it Description:} [Standard] Boolean parameter specifying if atomic forces are to be relaxed.


{\it Possible values:} A boolean value (true or false)
\item {\it Parameter name:} {\tt ION RELAX FLAGS FILE}
\phantomsection\label{parameters:Geometry/Optimization/ION RELAX FLAGS FILE}
\label{parameters:Geometry/Optimization/ION_20RELAX_20FLAGS_20FILE}


\index[prmindex]{ION RELAX FLAGS FILE}
\index[prmindexfull]{Geometry!Optimization!ION RELAX FLAGS FILE}


{\it Default:} 


{\it Description:} [Standard] File specifying the permission flags (1-free to move, 0-fixed) for the 3-coordinate directions and for all atoms. File format (example for two atoms with atom 1 fixed and atom 2 free): 0 0 0 (row1), 1 1 1 (row2).


{\it Possible values:} Any string
\item {\it Parameter name:} {\tt NON SELF CONSISTENT FORCE}
\phantomsection\label{parameters:Geometry/Optimization/NON SELF CONSISTENT FORCE}
\label{parameters:Geometry/Optimization/NON_20SELF_20CONSISTENT_20FORCE}


\index[prmindex]{NON SELF CONSISTENT FORCE}
\index[prmindexfull]{Geometry!Optimization!NON SELF CONSISTENT FORCE}


{\it Default:} false


{\it Description:} [Developer] Boolean parameter specifying whether to include the force contributions arising out of non self-consistency in the Kohn-Sham ground-state calculation. Currently non self-consistent force computation is still in experimental phase. The default option is false.


{\it Possible values:} A boolean value (true or false)
\item {\it Parameter name:} {\tt STRESS TOL}
\phantomsection\label{parameters:Geometry/Optimization/STRESS TOL}
\label{parameters:Geometry/Optimization/STRESS_20TOL}


\index[prmindex]{STRESS TOL}
\index[prmindexfull]{Geometry!Optimization!STRESS TOL}


{\it Default:} 1e-6


{\it Description:} [Standard] Sets the tolerance of the cell stress (in a.u.) during cell-relaxation.


{\it Possible values:} A floating point number $v$ such that $0 \leq v \leq 1$
\end{itemize}

\subsection{Parameters in section \tt Parallelization}
\label{parameters:Parallelization}

\begin{itemize}
\item {\it Parameter name:} {\tt NPBAND}
\phantomsection\label{parameters:Parallelization/NPBAND}


\index[prmindex]{NPBAND}
\index[prmindexfull]{Parallelization!NPBAND}


{\it Default:} 1


{\it Description:} [Standard] Number of groups of MPI tasks across which the work load of the bands is parallelised. NPKPT times NPBAND must be a divisor of total number of MPI tasks. Further, NPBAND must be less than or equal to NUMBER OF KOHN-SHAM WAVEFUNCTIONS.


{\it Possible values:} An integer $n$ such that $1\leq n \leq 2147483647$
\item {\it Parameter name:} {\tt NPKPT}
\phantomsection\label{parameters:Parallelization/NPKPT}


\index[prmindex]{NPKPT}
\index[prmindexfull]{Parallelization!NPKPT}


{\it Default:} 1


{\it Description:} [Standard] Number of groups of MPI tasks across which the work load of the irreducible k-points is parallelised. NPKPT times NPBAND must be a divisor of total number of MPI tasks. Further, NPKPT must be less than or equal to the number of irreducible k-points.


{\it Possible values:} An integer $n$ such that $1\leq n \leq 2147483647$
\end{itemize}

\subsection{Parameters in section \tt Poisson problem parameters}
\label{parameters:Poisson_20problem_20parameters}

\begin{itemize}
\item {\it Parameter name:} {\tt MAXIMUM ITERATIONS}
\phantomsection\label{parameters:Poisson problem parameters/MAXIMUM ITERATIONS}
\label{parameters:Poisson_20problem_20parameters/MAXIMUM_20ITERATIONS}


\index[prmindex]{MAXIMUM ITERATIONS}
\index[prmindexfull]{Poisson problem parameters!MAXIMUM ITERATIONS}


{\it Default:} 5000


{\it Description:} [Advanced] Maximum number of iterations to be allowed for Poisson problem convergence.


{\it Possible values:} An integer $n$ such that $0\leq n \leq 20000$
\item {\it Parameter name:} {\tt TOLERANCE}
\phantomsection\label{parameters:Poisson problem parameters/TOLERANCE}
\label{parameters:Poisson_20problem_20parameters/TOLERANCE}


\index[prmindex]{TOLERANCE}
\index[prmindexfull]{Poisson problem parameters!TOLERANCE}


{\it Default:} 1e-12


{\it Description:} [Advanced] Relative tolerance as stopping criterion for Poisson problem convergence.


{\it Possible values:} A floating point number $v$ such that $0 \leq v \leq 1$
\end{itemize}

\subsection{Parameters in section \tt SCF parameters}
\label{parameters:SCF_20parameters}

\begin{itemize}
\item {\it Parameter name:} {\tt ANDERSON SCHEME MIXING HISTORY}
\phantomsection\label{parameters:SCF parameters/ANDERSON SCHEME MIXING HISTORY}
\label{parameters:SCF_20parameters/ANDERSON_20SCHEME_20MIXING_20HISTORY}


\index[prmindex]{ANDERSON SCHEME MIXING HISTORY}
\index[prmindexfull]{SCF parameters!ANDERSON SCHEME MIXING HISTORY}


{\it Default:} 10


{\it Description:} [Standard] Number of SCF iteration history to be considered for mixing the electron-density using Anderson mixing scheme. For metallic systems, a mixing history larger than the default value provides better scf convergence.


{\it Possible values:} An integer $n$ such that $1\leq n \leq 1000$
\item {\it Parameter name:} {\tt ANDERSON SCHEME MIXING PARAMETER}
\phantomsection\label{parameters:SCF parameters/ANDERSON SCHEME MIXING PARAMETER}
\label{parameters:SCF_20parameters/ANDERSON_20SCHEME_20MIXING_20PARAMETER}


\index[prmindex]{ANDERSON SCHEME MIXING PARAMETER}
\index[prmindexfull]{SCF parameters!ANDERSON SCHEME MIXING PARAMETER}


{\it Default:} 0.5


{\it Description:} [Standard] Mixing parameter to be used in Anderson scheme.


{\it Possible values:} A floating point number $v$ such that $0 \leq v \leq 1$
\item {\it Parameter name:} {\tt COMPUTE ENERGY EACH ITER}
\phantomsection\label{parameters:SCF parameters/COMPUTE ENERGY EACH ITER}
\label{parameters:SCF_20parameters/COMPUTE_20ENERGY_20EACH_20ITER}


\index[prmindex]{COMPUTE ENERGY EACH ITER}
\index[prmindexfull]{SCF parameters!COMPUTE ENERGY EACH ITER}


{\it Default:} true


{\it Description:} [Advanced] Boolean parameter specifying whether to compute the total energy at the end of every SCF. Setting it to false can lead to some computational time savings.


{\it Possible values:} A boolean value (true or false)
\item {\it Parameter name:} {\tt MAXIMUM ITERATIONS}
\phantomsection\label{parameters:SCF parameters/MAXIMUM ITERATIONS}
\label{parameters:SCF_20parameters/MAXIMUM_20ITERATIONS}


\index[prmindex]{MAXIMUM ITERATIONS}
\index[prmindexfull]{SCF parameters!MAXIMUM ITERATIONS}


{\it Default:} 100


{\it Description:} [Standard] Maximum number of iterations to be allowed for SCF convergence


{\it Possible values:} An integer $n$ such that $1\leq n \leq 1000$
\item {\it Parameter name:} {\tt STARTING WFC}
\phantomsection\label{parameters:SCF parameters/STARTING WFC}
\label{parameters:SCF_20parameters/STARTING_20WFC}


\index[prmindex]{STARTING WFC}
\index[prmindexfull]{SCF parameters!STARTING WFC}


{\it Default:} RANDOM


{\it Description:} [Standard] Sets the type of the starting Kohn-Sham wavefunctions guess: Atomic(Superposition of single atom atomic orbitals. Atom types for which atomic orbitals are not available, random wavefunctions are taken. Currently, atomic orbitals data is not available for all atoms.), Random(The starting guess for all wavefunctions are taken to be random). Default: RANDOM.


{\it Possible values:} Any one of ATOMIC, RANDOM
\item {\it Parameter name:} {\tt TEMPERATURE}
\phantomsection\label{parameters:SCF parameters/TEMPERATURE}
\label{parameters:SCF_20parameters/TEMPERATURE}


\index[prmindex]{TEMPERATURE}
\index[prmindexfull]{SCF parameters!TEMPERATURE}


{\it Default:} 500.0


{\it Description:} [Standard] Fermi-Dirac smearing temperature (in Kelvin).


{\it Possible values:} A floating point number $v$ such that $1e-05 \leq v \leq \text{MAX\_DOUBLE}$
\item {\it Parameter name:} {\tt TOLERANCE}
\phantomsection\label{parameters:SCF parameters/TOLERANCE}
\label{parameters:SCF_20parameters/TOLERANCE}


\index[prmindex]{TOLERANCE}
\index[prmindexfull]{SCF parameters!TOLERANCE}


{\it Default:} 1e-06


{\it Description:} [Standard] SCF iterations stopping tolerance in terms of $L_2$ norm of the electron-density difference between two successive iterations. CAUTION: A tolerance close to 1e-7 or lower can deteriorate the SCF convergence due to the round-off error accumulation.


{\it Possible values:} A floating point number $v$ such that $1e-12 \leq v \leq 1$
\end{itemize}



\subsection{Parameters in section \tt SCF parameters/Eigen-solver parameters}
\label{parameters:SCF_20parameters/Eigen_2dsolver_20parameters}

\begin{itemize}
\item {\it Parameter name:} {\tt BATCH GEMM}
\phantomsection\label{parameters:SCF parameters/Eigen_2dsolver parameters/BATCH GEMM}
\label{parameters:SCF_20parameters/Eigen_2dsolver_20parameters/BATCH_20GEMM}


\index[prmindex]{BATCH GEMM}
\index[prmindexfull]{SCF parameters!Eigen-solver parameters!BATCH GEMM}


{\it Default:} true


{\it Description:} [Advanced] Boolean parameter specifying whether to use gemm batch blas routines to perform matrix-matrix multiplication operations with groups of matrices, processing a number of groups at once using threads instead of the standard serial route. CAUTION: gemm batch blas routines will only be activated if the WFC BLOCK SIZE is less than 1000, and only if intel mkl blas library is linked with the dealii installation. Default option is true.


{\it Possible values:} A boolean value (true or false)
\item {\it Parameter name:} {\tt CHEBYSHEV FILTER TOLERANCE}
\phantomsection\label{parameters:SCF parameters/Eigen_2dsolver parameters/CHEBYSHEV FILTER TOLERANCE}
\label{parameters:SCF_20parameters/Eigen_2dsolver_20parameters/CHEBYSHEV_20FILTER_20TOLERANCE}


\index[prmindex]{CHEBYSHEV FILTER TOLERANCE}
\index[prmindexfull]{SCF parameters!Eigen-solver parameters!CHEBYSHEV FILTER TOLERANCE}


{\it Default:} 1e-02


{\it Description:} [Advanced] Parameter specifying the accuracy of the occupied eigenvectors close to the Fermi-energy computed using Chebyshev filtering subspace iteration procedure. Default value is sufficient for most purposes


{\it Possible values:} A floating point number $v$ such that $1e-10 \leq v \leq \text{MAX\_DOUBLE}$
\item {\it Parameter name:} {\tt CHEBYSHEV POLYNOMIAL DEGREE}
\phantomsection\label{parameters:SCF parameters/Eigen_2dsolver parameters/CHEBYSHEV POLYNOMIAL DEGREE}
\label{parameters:SCF_20parameters/Eigen_2dsolver_20parameters/CHEBYSHEV_20POLYNOMIAL_20DEGREE}


\index[prmindex]{CHEBYSHEV POLYNOMIAL DEGREE}
\index[prmindexfull]{SCF parameters!Eigen-solver parameters!CHEBYSHEV POLYNOMIAL DEGREE}


{\it Default:} 0


{\it Description:} [Advanced] Chebyshev polynomial degree to be employed for the Chebyshev filtering subspace iteration procedure to dampen the unwanted spectrum of the Kohn-Sham Hamiltonian. If set to 0, a default value depending on the upper bound of the eigen-spectrum is used. See Phani Motamarri et.al., J. Comp. Phys. 253, 308-343 (2013).


{\it Possible values:} An integer $n$ such that $0\leq n \leq 2000$
\item {\it Parameter name:} {\tt ENABLE SUBSPACE ROT PGS OPT}
\phantomsection\label{parameters:SCF parameters/Eigen_2dsolver parameters/ENABLE SUBSPACE ROT PGS OPT}
\label{parameters:SCF_20parameters/Eigen_2dsolver_20parameters/ENABLE_20SUBSPACE_20ROT_20PGS_20OPT}


\index[prmindex]{ENABLE SUBSPACE ROT PGS OPT}
\index[prmindexfull]{SCF parameters!Eigen-solver parameters!ENABLE SUBSPACE ROT PGS OPT}


{\it Default:} true


{\it Description:} [Developer] Turns on subspace rotation optimization for Pseudo-Gram-Schimdt orthogonalization. Default option is true.


{\it Possible values:} A boolean value (true or false)
\item {\it Parameter name:} {\tt ENABLE SWITCH TO GS}
\phantomsection\label{parameters:SCF parameters/Eigen_2dsolver parameters/ENABLE SWITCH TO GS}
\label{parameters:SCF_20parameters/Eigen_2dsolver_20parameters/ENABLE_20SWITCH_20TO_20GS}


\index[prmindex]{ENABLE SWITCH TO GS}
\index[prmindexfull]{SCF parameters!Eigen-solver parameters!ENABLE SWITCH TO GS}


{\it Default:} true


{\it Description:} [Developer] Controls automatic switching to Gram-Schimdt orthogonalization if Lowden Orthogonalization or Pseudo-Gram-Schimdt orthogonalization are unstable. Default option is true.


{\it Possible values:} A boolean value (true or false)
\item {\it Parameter name:} {\tt LOWER BOUND UNWANTED FRAC UPPER}
\phantomsection\label{parameters:SCF parameters/Eigen_2dsolver parameters/LOWER BOUND UNWANTED FRAC UPPER}
\label{parameters:SCF_20parameters/Eigen_2dsolver_20parameters/LOWER_20BOUND_20UNWANTED_20FRAC_20UPPER}


\index[prmindex]{LOWER BOUND UNWANTED FRAC UPPER}
\index[prmindexfull]{SCF parameters!Eigen-solver parameters!LOWER BOUND UNWANTED FRAC UPPER}


{\it Default:} 0


{\it Description:} [Developer] The value of the fraction of the upper bound of the unwanted spectrum, the lower bound of the unwanted spectrum will be set. Default value is 0.


{\it Possible values:} A floating point number $v$ such that $0 \leq v \leq 1$
\item {\it Parameter name:} {\tt LOWER BOUND WANTED SPECTRUM}
\phantomsection\label{parameters:SCF parameters/Eigen_2dsolver parameters/LOWER BOUND WANTED SPECTRUM}
\label{parameters:SCF_20parameters/Eigen_2dsolver_20parameters/LOWER_20BOUND_20WANTED_20SPECTRUM}


\index[prmindex]{LOWER BOUND WANTED SPECTRUM}
\index[prmindexfull]{SCF parameters!Eigen-solver parameters!LOWER BOUND WANTED SPECTRUM}


{\it Default:} -10.0


{\it Description:} [Developer] The lower bound of the wanted eigen spectrum. It is only used for the first iteration of the Chebyshev filtered subspace iteration procedure. A rough estimate based on single atom eigen values can be used here. Default value is good enough for most problems.


{\it Possible values:} A floating point number $v$ such that $-\text{MAX\_DOUBLE} \leq v \leq \text{MAX\_DOUBLE}$
\item {\it Parameter name:} {\tt NUMBER OF KOHN-SHAM WAVEFUNCTIONS}
\phantomsection\label{parameters:SCF parameters/Eigen_2dsolver parameters/NUMBER OF KOHN_2dSHAM WAVEFUNCTIONS}
\label{parameters:SCF_20parameters/Eigen_2dsolver_20parameters/NUMBER_20OF_20KOHN_2dSHAM_20WAVEFUNCTIONS}


\index[prmindex]{NUMBER OF KOHN-SHAM WAVEFUNCTIONS}
\index[prmindexfull]{SCF parameters!Eigen-solver parameters!NUMBER OF KOHN-SHAM WAVEFUNCTIONS}


{\it Default:} 10


{\it Description:} [Standard] Number of Kohn-Sham wavefunctions to be computed. For spin-polarized calculations, this parameter denotes the number of Kohn-Sham wavefunctions to be computed for each spin. A recommended value for this parameter is to set it to N/2+Nb where N is the number of electrons. Use Nb to be 10-20 percent of N/2 for insulators and for metals use Nb to be 20 percent of N/2. If 10-20 percent of N/2 is less than 10 wavefunctions, set Nb to be atleast 10.


{\it Possible values:} An integer $n$ such that $0\leq n \leq 2147483647$
\item {\it Parameter name:} {\tt ORTHOGONALIZATION TYPE}
\phantomsection\label{parameters:SCF parameters/Eigen_2dsolver parameters/ORTHOGONALIZATION TYPE}
\label{parameters:SCF_20parameters/Eigen_2dsolver_20parameters/ORTHOGONALIZATION_20TYPE}


\index[prmindex]{ORTHOGONALIZATION TYPE}
\index[prmindexfull]{SCF parameters!Eigen-solver parameters!ORTHOGONALIZATION TYPE}


{\it Default:} Auto


{\it Description:} [Advanced] Parameter specifying the type of orthogonalization to be used: GS(Gram-Schmidt Orthogonalization using SLEPc library), LW(Lowden Orthogonalization implemented using LAPACK/BLAS routines, extension to use ScaLAPACK library not implemented yet), PGS(Pseudo-Gram-Schmidt Orthogonalization: if dealii library is compiled with ScaLAPACK and if you are using the real executable, parallel ScaLAPACK functions are used, otherwise serial LAPACK functions are used.) Auto is the default option, which chooses GS for all-electron case and PGS for pseudopotential case.


{\it Possible values:} Any one of GS, LW, PGS, Auto
\item {\it Parameter name:} {\tt SCALAPACKPROCS}
\phantomsection\label{parameters:SCF parameters/Eigen_2dsolver parameters/SCALAPACKPROCS}
\label{parameters:SCF_20parameters/Eigen_2dsolver_20parameters/SCALAPACKPROCS}


\index[prmindex]{SCALAPACKPROCS}
\index[prmindexfull]{SCF parameters!Eigen-solver parameters!SCALAPACKPROCS}


{\it Default:} 0


{\it Description:} [Advanced] Uses a processor grid of SCALAPACKPROCS times SCALAPACKPROCS for parallel distribution of the subspace projected matrix in the Rayleigh-Ritz step and the overlap matrix in the Pseudo-Gram-Schmidt step. Default value is 0 for which a thumb rule is used (see http://netlib.org/scalapack/slug/node106.html). This parameter is only used if dealii library is compiled with ScaLAPACK.


{\it Possible values:} An integer $n$ such that $0\leq n \leq 300$
\item {\it Parameter name:} {\tt SPECTRUM SPLIT CORE EIGENSTATES}
\phantomsection\label{parameters:SCF parameters/Eigen_2dsolver parameters/SPECTRUM SPLIT CORE EIGENSTATES}
\label{parameters:SCF_20parameters/Eigen_2dsolver_20parameters/SPECTRUM_20SPLIT_20CORE_20EIGENSTATES}


\index[prmindex]{SPECTRUM SPLIT CORE EIGENSTATES}
\index[prmindexfull]{SCF parameters!Eigen-solver parameters!SPECTRUM SPLIT CORE EIGENSTATES}


{\it Default:} 0


{\it Description:} [Advanced] Number of lowest Kohn-Sham eigenstates which should not be included in the Rayleigh-Ritz projection step.  In other words, only the higher eigenstates (Number of Kohn-Sham wavefunctions minus the specified core eigenstates) are used to compute projected Hamiltonian and subsequently diagonalization is done on the projected Hamiltonian corresponding to the higher eigenstates. This value is usually chosen to be the sum of the number of core eigenstates for each atom type multiplied by number of atoms of that type. This setting is recommended for large systems (greater than 5000 electrons). Default value is 0 i.e., no core eigenstates are excluded from the Rayleigh-Ritz projection step.


{\it Possible values:} An integer $n$ such that $0\leq n \leq 2147483647$
\item {\it Parameter name:} {\tt SUBSPACE ROT DOFS BLOCK SIZE}
\phantomsection\label{parameters:SCF parameters/Eigen_2dsolver parameters/SUBSPACE ROT DOFS BLOCK SIZE}
\label{parameters:SCF_20parameters/Eigen_2dsolver_20parameters/SUBSPACE_20ROT_20DOFS_20BLOCK_20SIZE}


\index[prmindex]{SUBSPACE ROT DOFS BLOCK SIZE}
\index[prmindexfull]{SCF parameters!Eigen-solver parameters!SUBSPACE ROT DOFS BLOCK SIZE}


{\it Default:} 2000


{\it Description:} [Developer] This block size is used for memory optimization purposes in subspace rotation step in Pseudo-Gram-Schmidt orthogonalization and Rayleigh-Ritz steps. This optimization is only activated if dealii library is compiled with ScaLAPACK. Default value is 2000.


{\it Possible values:} An integer $n$ such that $1\leq n \leq 2147483647$
\item {\it Parameter name:} {\tt WFC BLOCK SIZE}
\phantomsection\label{parameters:SCF parameters/Eigen_2dsolver parameters/WFC BLOCK SIZE}
\label{parameters:SCF_20parameters/Eigen_2dsolver_20parameters/WFC_20BLOCK_20SIZE}


\index[prmindex]{WFC BLOCK SIZE}
\index[prmindexfull]{SCF parameters!Eigen-solver parameters!WFC BLOCK SIZE}


{\it Default:} 400


{\it Description:} [Advanced] Chebyshev filtering procedure involves the matrix-matrix multiplication where one matrix corresponds to the discretized Hamiltonian and the other matrix corresponds to the wavefunction matrix. The matrix-matrix multiplication is accomplished in a loop over the number of blocks of the wavefunction matrix to reduce the memory footprint of the code. This parameter specifies the block size of the wavefunction matrix to be used in the matrix-matrix multiplication. The optimum value is dependent on the computing architecture. The same block size also used for memory optimization purposes in the orthogonalization and Rayleigh-Ritz steps. The memory optimization part is activated only if dealii library is compiled with ScaLAPACK. For optimum work sharing during band parallelization (NPBAND > 1), we recommend adjusting WFC BLOCK SIZE and NUMBER OF KOHN-SHAM WAVEFUNCTIONS such that NUMBER OF KOHN-SHAM WAVEFUNCTIONS/NPBAND/WFC BLOCK SIZE equals an integer value. Default value is 400.


{\it Possible values:} An integer $n$ such that $1\leq n \leq 2147483647$
\end{itemize}
